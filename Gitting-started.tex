GITHUB - #Started April 30th 2014
#1. PERSONAL ACCOUNT and FIRST TIME CONFIGURATION COMMENTS PER MACHING:
git config --global user.name "juliakbaum"   # Sets the default name for git to use when you commit
git config --global user.email "juliakbaum@gmail.com" # Sets the default email for git to use when you commit
git config --global core.editor aquamacs            ## Sets the default editor in terminal. Ignore this for now
See also: https://help.github.com/articles/set-up-git

#2. BASIC MANEUVERS / NAVIGATING GIT
-Git icon button in top left - is the MAIN SCREEN Note: Repositories which I am collaborating are only viewable on this screen

-Username/photo icon in upper right shows dashboard for:
	1) Contributions: Lists the repositories that you have recently worked in, and your Github activity over the last year
	2) Repositories: Shows your private and public repositories, and the repos you have forked from other members 
	3) Public activity: Shows your work on Github - what have you forked, pulled, merged?


#3. BAUM LAB ORGANISATION

-To use Github within the Baum lab organisation, contact James or Julia to be added as a member. For any project in the Baum lab, we will create a team that allows multiple users to create and edit repositories. 

-Each team will be for a specific project/manuscript.

-The read/write permissions in an organisation can get confusing, but to keep things simple we will allow all members to read and write (fork, commit and push to the Baum lab project repo). This may result in a barrage of emails from Github going to every project member whenever a pull request is made, but let's see how it goes for now.


#3. CREATING REPO: 
See also: https://help.github.com/articles/create-a-repo

### FOR TEAMS - we will do this when multiple members want to collaborate on the code and manuscript for one project/manuscript

For our purposes, the easiest way to do this is to create a repo on Github, fork it, and then clone onto your desktop.

	1) Go to github.com/baumlab/teams/teamname
	2) Click "Add repositories" to create new repo

#4 FORKING THE BAUM LAB REPO

See also: https://help.github.com/articles/fork-a-repo

Now every member of the team will have access to the repository. Each member, and you, should go to baumlab/projectname and click "Fork" at the top right of the screen.

Now your personal Github account has forked the new repository. The next step is to clone this onto your local machine.

Open up terminal, navigate to the folder that you want to keep the repo in, then type:

git clone https://github.com/username/reponame.git

Now your computer will have a local version of the repo files - the master branch. When you make changes on your computer, you will commit and push the changes to your personal github server.


#5 MAKING COMMITS AND PUSHING TO GITHUB

Now you work. And make edits. And push to github. This section is just for you - you won't be able to make changes to the team repo yet.

To add files to the repo:

git add filename

To commit the changes to the repo:

git commit -m "This is my first commit"

Now your local repository - the master branch - has some changes that aren't up to date with the Github server (the origin branch). You can check this using:

git status

Here you will either see "Your branch is up to date" and no changes have been made. Or, you will see red files that are "ahead" of the origin, and are called uncommitted changes. You need to push these up to github:

git push origin master

This command sends the commits from "master" to "origin" (from your computer to Github).

#6 SENDING EDITS TO THE BAUM LAB REPO
See also: https://help.github.com/articles/fork-a-repo

In our github hierarchy, we are one step away from the original, shared repository. Currently, your local machine is linked to your personal github account, that was originally forked from the Baum lab team repo. We need to connect back to the Baum lab, and the final step is called the "upstream". We need to tell Github where upstream actually is.

So, in the master directory (your repo on your machine) go to terminal and type:

git remote add upstream https://github.com/baumlab/reponame.git

This tells your personal github account that the original repository is located at Baum lab. We then need to fetch any changes that might have been made to the original:

git fetch upstream
git merge upstream/master

This will change your local files to match those in the Baum lab repository.

Now, the important part - changing the files in the Baum lab repo. To do this we need to send a pull request: https://help.github.com/articles/using-pull-requests

On github.com, navigate to 



#4. Commit a file e.g. README. A commit is essentially a snapshot of all the files in your project at a particular point in time:
git add README 			#Stages your README file, adding it to the list of files to be committed
git commit -m ‘comments’	#Commits your files, adding the comments in quotes - 
Note: At this point it is still in the local repository
To connect your local repo to your GitHub account, you need to set a remote for your repository and push your commits to it: 

#5. WORKING COLLABORATIVELY with a STUDENT
# SCENARIO 1 - STUDENT has set up the REPO with their account & it is PRIVATE

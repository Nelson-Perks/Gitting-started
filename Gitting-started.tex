GITHUB - #Started April 30th 2014
#1. PERSONAL ACCOUNT and FIRST TIME CONFIGURATION COMMENTS PER MACHING:
git config --global user.name "juliakbaum"   # Sets the default name for git to use when you commit
git config --global user.email "juliakbaum@gmail.com" # Sets the default email for git to use when you commit
git config --global core.editor aquamacs
See also: https://help.github.com/articles/set-up-git

#2. BASIC MANEUVERS / NAVIGATING GIT
-Git icon button in top left - is the MAIN SCREEN Note: Repositories which I am collaborating are only viewable on this screen

-Username/photo icon in upper right shows dashboard for:
	1) Contributions: 
	2) Repositories: NOTE that this only shows my repositories **Thus, it shows my forked ‘Cocos_Paper’ repo, but not Easton’s erwhite1/Cocos_Paper repo to which I am contributing (via this fork) - See above to see these. 
	3) Public activity:

#3. CREATING REPO: 
See also: https://help.github.com/articles/create-a-repo
Note that I did this by making a directory on my local machine, then making a repo on github with same name, then linking. In actuality,
one only has to do one OR the other, and then push the one to the other; 
git init					#Sets up the necessary Git files
Initialized empty Git repository in the directory you’ve specified e.g. /Users/you/Hello-World/.git/

#4. Commit a file e.g. README. A commit is essentially a snapshot of all the files in your project at a particular point in time:
git add README 			#Stages your README file, adding it to the list of files to be committed
git commit -m ‘comments’	#Commits your files, adding the comments in quotes - 
Note: At this point it is still in the local repository
To connect your local repo to your GitHub account, you need to set a remote for your repository and push your commits to it: 

#5. WORKING COLLABORATIVELY with a STUDENT
# SCENARIO 1 - STUDENT has set up the REPO with their account & it is PRIVATE
